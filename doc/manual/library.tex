
\chapter{The LBJava Library}

The LBJava programming framework is supported by a library of interfaces,
learning algorithms, and implementations of the building blocks described in
Chapter \ref{chapter:language}.  This chapter gives a general overview of each
of those codes.  More detailed usage descriptions can be found in the online
Javadoc at {\tt http://cogcomp.cs.illinois.edu/software/doc/LBJava/apidocs}. \\

The most commonly used features of the library are found in the following five packages.  {\tt lbjava.classify}
contains classes related to features and classification.  {\tt lbjava.learn}
contains learner implementations and supporting classes.  {\tt lbjava.infer}
contains inference algorithm implementations and internal representations for
constraints and inference structures.  {\tt lbjava.parse} contains the {\tt
Parser} interface and some general purpose internal representation classes.
Finally, {\tt lbjava.nlp} contains some basic natural language processing
internal representations and parsing routines.  In the future, we plan to
expand this library, adding more varieties of learners and domain specific
parsers and internal representations.

\section{{\tt lbjava.classify}}

The most important class in LBJava's library is {\tt lbjava.classify.Classifier}.
This abstract class is the interface through which the application accesses
the classifiers defined in the LBJava source file.  However, the programmer
should, in general, only have need to become familiar with a few of the
methods defined there. \\

One other class that may be of broad interest is the {\tt
lbjava.classify.TestDiscrete} class (discussed in Section
\ref{subsection:testDiscrete}), which can automate the performance evaluation
of a discrete learning classifier on a labeled test set.  The other classes in
this package are designed mainly for internal use by LBJava's compiler and can be
safely ignored by the casual user.  More advanced users who writes their own
learners or inference algorithms in the application, for instance, will need
to become familiar with them.

\subsection{{\tt lbjava.classify.Classifier}} \label{subsection:classifier}
Every classifier declaration in an LBJava source file is translated by the
compiler into a Java class that extends this class.  When the programmer wants
to call a classifier in the application, he creates an object of his
classifier's class using its zero argument constructor and calls an
appropriate method on that object.  The appropriate method will most likely be
one of the following four methods:

\begin{list}{}{}
\item[{\tt String discreteValue(Object)}:] ~\\
This method will only be overridden in the classifier's implementation if its
feature return type is {\tt discrete}.  Its return value is the value of the
single feature this classifier returns.

\item[{\tt double realValue(Object)}:] ~\\
This method will only be overridden in the classifier's implementation if its
feature return type is {\tt real}.  Its return value is the value of the
single feature this classifier returns.

\item[{\tt String[] discreteValueArray(Object)}:] ~\\
This method will only be overridden in the classifier's implementation if its
feature return type is {\tt discrete[]}.  Its return value contains the values
of all the features this classifier returns.

\item[{\tt double[] realValueArray(Object)}:] ~\\
This method will only be overridden in the classifier's implementation if its
feature return type is {\tt real[]}.  Its return value contains the values of
all the features this classifier returns.
\end{list}

There is no method similar to the four above for accessing the values of
features produced by a feature generator, since those values are meaningless
without their associated names.  When the programmer wants access to the
actual features produced by any classifier (not just feature generators), the
following non-static method is used.  Note, however, that the main purpose of
this method is for internal use by the compiler.\footnote{One circumstance
where the programmer may be interested in this method is to print out the {\tt
String} representation of the returned {\tt FeatureVector}.}

\begin{list}{}{}
\item[{\tt FeatureVector classify(Object)}:] ~\\
This method is overridden in every classifier implementation generated by the
LBJava compiler.  It returns a {\tt FeatureVector} which may be iterated through
to access individual features (see Section \ref{subsection:featureVector}).
\end{list}

Every classifier implementation generated by the compiler overrides the
following non-static member methods as well.  They provide type information
about the implemented classifier.

\begin{list}{}{}
\item[{\tt String getInputType()}:] ~\\
This method returns a {\tt String} containing the fully qualified name of the
class this classifier expects as input.

\item[{\tt String getOutputType()}:] ~\\
This method returns a {\tt String} containing the feature return type of this
classifier.  If the classifier is {\tt discrete} and contains a list of
allowable values, it will not appear in the output of this method.

\item[{\tt String[] allowableValues()}:] ~\\
If the classifier is {\tt discrete} and contains a list of allowable values,
that list will be returned by this method.  Otherwise, an array of length zero
is returned.  Learners that require a particular number of allowable values
may return an array filled with {\tt "*"} whose length indicates that number.
\end{list}

Finally, class {\tt Classifier} provides a simple static method for testing
the agreement of two classifiers.  It's convenient, for instance, when testing
the performance of a learned classifier against an oracle classifier.

\begin{list}{}{}
\item[{\tt double test(Classifier, Classifier, Object[])}:] ~\\
This static method returns the fraction of objects in the third argument that
produced the same classifications from the two argument {\tt Classifier}s.
\end{list}

There are several other methods of this class described in the Javadoc
documentation.  They are omitted here since the programmer is not expected to
need them.

\subsection{{\tt lbjava.classify.Feature}} \label{subsection:feature}
This abstract class is part of the representation of the value produced by a
classifier.  In particular, the name of a feature, but not its value, is
stored here.  Classes derived from this class (described below) provide
storage for the value of the feature.  This class exists mainly for internal
use by the LBJava compiler, and most programmers will not need to be familiar
with it.

\begin{list}{}{}
\item[{\tt lbjava.classify.DiscreteFeature}:] ~\\
The value of a feature returned by a {\tt discrete} classifier is stored as a
{\tt String} in objects of this class.

\item[{\tt lbjava.classify.DiscreteArrayFeature}:] ~\\
The {\tt String} value of a feature returned by a {\tt discrete[]} classifier
as well as its integer index into the array are stored in objects of this
class.

\item[{\tt lbjava.classify.RealFeature}:] ~\\
The value of a feature returned by a {\tt real} classifier is stored as a {\tt
double} in objects of this class.

\item[{\tt lbjava.classify.RealArrayFeature}:] ~\\
The {\tt double} value of a feature returned by a {\tt real[]} classifier as
well as its integer index into the array are stored in objects of this class.
\end{list}

\subsection{{\tt lbjava.classify.FeatureVector}}\label{subsection:featureVector}
{\tt FeatureVector} is a linked-list-style container which stores features
that function as labels separately from other features.  It contains methods
for iterating through the features and labels and adding more of either.  Its
main function is as the return value of the {\tt Classifier\#classify(Object)}
method which is used internally by the LBJava compiler (see Section
\ref{subsection:classifier}).  Most programmers will not need to become
intimately familiar with this class.

\subsection{{\tt lbjava.classify.Score}}
This class represents the {\tt double} score produced by a discrete learning
classifier is association with one of its {\tt String} prediction values.
Both items are stored in an object of this class.  This class is used
internally by LBJava's inference infrastructure, which will interpret the score
as an indication of how much the learning classifier prefers the associated
prediction value, higher scores indicating more preference.

\subsection{{\tt lbjava.classify.ScoreSet}} \label{subsection:scoreSet}
This is another class used internally by LBJava's inference infrastructure.  An
object of this class is intended to contain one {\tt Score} for each possible
prediction value a learning classifier is capable of returning.

\subsection{{\tt lbjava.classify.ValueComparer}}
This simple class derived from {\tt Classifier} is used to convert a
multi-value {\tt discrete} classifier into a Boolean classifier that returns
{\tt true} if and only if the multi-valued classifier evaluated to a
particular value.  {\tt ValueComparer} is used internally by {\tt
SparseNetworkLearner} (see Section \ref{subsection:SNL}).

\subsection{Vector Returners} \label{subsection:vectorReturners}
The classes {\tt lbjava.classify.FeatureVectorReturner} and \\
{\tt lbjava.classify.LabelVectorReturner} are used internally by the LBJava
compiler to help implement the training procedure when the programmer
specifies multiple training rounds (see Section \ref{subsection:LCE}).  A
feature vector returner is substituted as the learning classifier's feature
extraction classifier, and a label vector returner is substituted as the
learning classifier's labeler (see Section \ref{subsection:learner} to see how
this substitution is performed).  Each of them then expects the object
received as input by the learning classifier to be a {\tt FeatureVector},
which is not normally the case.  However, as will be described in Section
\ref{subsection:FVP}, the programmer may still be interested in these classes
if he wishes to continue training a learning classifier for additional rounds
on the same data without incurring the costs of performing feature extraction.

\subsection{{\tt lbjava.classify.TestDiscrete}} \label{subsection:testDiscrete}
This class can be quite useful to quickly evaluate the performance of a newly
learned classifier on labeled testing data.  It operates either as a
stand-alone program or as a class that may be imported into an application for
more tailored use.  In either case, it will automatically compute accuracy,
precision, recall, and F1 scores for the learning classifier in question. \\

To use this class inside an application, simply instantiate an object of it
using the no-argument constructor.  Lets call this object {\tt tester}.  Then,
each time the learning classifier makes a prediction {\tt p} for an object
whose true label is {\tt l}, make the call
{\tt tester.reportPrediction(p, l)}.  Once all testing objects have been
processed, the {\tt printPerformance(java.io.PrintStream)} method may be used
print a table of results, or the programmer may make use of the various other
methods provided by this class to retrieve the computed statistics.  More
detailed usage of all these methods as well as the operation of this class as
a stand-alone program is available in the on-line Javadoc.

\section{{\tt lbjava.learn}}

The programmer will want to familiarize himself with most of the classes in
this package, in particular those that are derived from the abstract class
{\tt lbjava.learn.Learner}.  These are the learners that may be selected from
within an LBJava source file in association with a learning classifier expression
(see Section \ref{subsection:LCE}).

\subsection{{\tt lbjava.learn.Learner}} \label{subsection:learner}
{\tt Learner} is an abstract class extending the abstract class {\tt
Classifier} (see Section \ref{subsection:classifier}).  It acts as an
interface between learning classifiers defined in an LBJava source file and
applications that make on-line use of their learning capabilities.  The class
generated by the LBJava compiler when translating a learning classifier
expression will always indirectly extend this class. \\

In addition to the methods inherited from {\tt Classifier}, this class defines
the following non-static, learning related methods.  These are not the only
methods defined in class {\tt Learner}, and advanced users may be interested
in perusing the Javadoc for descriptions of other methods.

\begin{list}{}{}
\item[{\tt void learn(Object)}:] ~\\
The programmer may call this method at any time from within the application to
continue the training process given a single example object.  The most common
use of this method will be in conjunction with a supervised learning
algorithm, in which case, of course, the true label of the example object must
be accessible by the label classifier specified in the learning classifier
expression in the LBJava source file.  Note that changes made via this method
will not persist beyond the current execution of the application unless the
{\tt save()} method (discussed below) is invoked.

\item[{\tt void doneLearning()}:] ~\\
Some learning algorithms (usually primarily off-line learning algorithms) save
part of their computation until after all training objects have been
observed.  This method informs the learning algorithm that it is time to
perform that part of the computation.  When compile-time training is indicated
in a learning classifier expression, the LBJava compiler will call this method
after training is complete.  Similarly, the programmer who performs on-line
learning in his application may need to call this method as well, depending on
the learning algorithm.

\item[{\tt void forget()}:] ~\\
The user may call this method from the application to reinitialize the
learning classifier to the state at which it started before any training was
performed.  Note that changes made via this method will not persist beyond
the current execution of the application unless the {\tt save()} method
(discussed below) is invoked.

\item[{\tt void save()}:] ~\\
As described in Section \ref{section:declarations}, the changes made while
training a classifier on-line in the application are immediately visible
everywhere in the application.  These changes are not written back to disk
unless the {\tt save()} method is invoked.  Once this method is invoked,
changes that have been made from on-line learning will become visible to
subsequent executions of applications that invoke this learning
classifier.\footnote{Please note that the {\tt save()} method currently will
not work when the classifier's byte code is packed in a jar file.}

\item[{\tt lbjava.classify.ScoreSet scores(Object)}:] ~\\
This method is used internally by inference algorithms which interpret the
scores in the returned {\tt ScoreSet} (see Section \ref{subsection:scoreSet})
as indications of which predictions the learning classifier prefers and how
much they are preferred.

\item[{\tt lbjava.classify.Classifier getExtractor()}:] ~\\
This method gives access to the feature extraction classifier used by this
learning classifier.

\item[{\tt void setExtractor(lbjava.classify.Classifier)}:] ~\\
Use this method to change the feature extraction classifier used by this
learning classifier.  Note that this change will be remembered during
subsequent executions of the application if the {\tt save()} method (described
above) is later invoked.

\item[{\tt lbjava.classify.Classifier getLabeler()}:] ~\\
This method gives access to the classifier used by this learning classifier to
produce labels for supervised learning.

\item[{\tt void setLabeler(lbjava.classify.Classifier)}:] ~\\
Use this method to change the labeler used by this learning classifier.  Note
that this change will be remembered during subsequent executions of the
application if the {\tt save()} method (described above) is later invoked.

\item[{\tt void write(java.io.PrintStream)}:] ~\\
This abstract method must be overridden by each extending learner
implementation.  A learning classifier derived from such a learner may then
invoke this method to produce the learner's internal representation in text
form.  Invoking this method does \emph{not} make modifications to the
learner's internal representation visible to subsequent executions of
applications that invoke this learning classifier like the {\tt save()} method
does.
\end{list}

In addition, the following \emph{static} flag is declared in every learner
output by the LBJava compiler.

\begin{list}{}{}
\item[{\tt public static boolean isTraining:}] ~\\
The {\tt isTraining} variable can be used by the programmer to determine if
his learning classifier is currently being trained.  This ability may be
useful if, for instance, a feature extraction classifier for this learning
classifier needs to alter its behavior depending on the availability of
labeled training data.  The LBJava compiler will automatically set this flag {\tt
true} during offline training, and it will be initialized {\tt false} in any
application using the learning classifier.  So, it becomes the programmer's
responsibility to make sure it is set appropriately if any additional online
training is to be performed in the application.
\end{list}

\subsection{{\tt lbjava.learn.LinearThresholdUnit}} \label{subsection:LTU}
A linear threshold unit is a supervised, mistake driven learner for binary
classification.  The predictions made by such a learner are produced by
computing a score for a given example object and then comparing that score to
a predefined threshold.  While learning, if the prediction does not match the
label, the linear function that produced the score is updated.  Linear
threshold units form the basis of many other learning techniques. \\

Class {\tt LinearThresholdUnit} is an abstract class defining a basic API for
learners of this type.  A non-abstract class extending it need only provide
implementations of the following abstract methods.

\begin{list}{}{}
\item[{\tt void promote(Object)}:] ~\\
This method makes an appropriate modification to the linear function
when a mistake is made on a positive example (i.e., when the computed score
mistakenly fell below the predefined threshold).

\item[{\tt void demote(Object)}:] ~\\
This method makes an appropriate modification to the linear function
when a mistake is made on a negative example (i.e., when the computed score
mistakenly rose above the predefined threshold).
\end{list}

When a learning classifier expression (see Section \ref{subsection:LCE})
employs a learner derived from this class, the specified label producing
classifier must be defined as {\tt discrete} with a value list containing
exactly two values\footnote{See Section \ref{section:declarations} for more
information on value lists in feature return types.}.  The learner derived
from this class will then learn to produce a higher score when the correct
prediction is the second value in the value list.

\subsection{{\tt lbjava.learn.SparsePerceptron}} \label{subsection:perceptron}
This learner extends class {\tt LinearThresholdUnit} (see Section
\ref{subsection:LTU}).  It represents its linear function for score
computation as a vector of weights corresponding to features.  It has an
additive update rule, meaning that it promotes and demotes by treating the
collection of features associated with a training object as a vector and using
vector addition.  Finally, parameters such as its learning rate, threshold,
the thick separator, and others described in the online Javadoc can be
configured by the user.

\subsection{{\tt lbjava.learn.SparseAveragedPerceptron}}
Extended from {\tt SparsePerceptron} (see Section
\ref{subsection:perceptron}), this learner computes an approximation of voted
Perceptron by averaging the weight vectors obtained after processing each
training example.  Its configurable parameters are the same as those of {\tt
SparsePerceptron}, and, in particular, using this algorithm in conjunction
with a positive thickness for the thick separator can be particularly
effective.

\subsection{{\tt lbjava.learn.SparseWinnow}}
This learner extends class {\tt LinearThresholdUnit} (see Section
\ref{subsection:LTU}).  It represents its linear function for score
computation as a vector of weights corresponding to features.  It has a
multiplicative update rule, meaning that it promotes and demotes by
multiplying an individual weight in the weight vector by a function of the
corresponding feature.  Finally, parameters such as its learning rates,
threshold, and others described in the online Javadoc can be configured by the
user.

\subsection{{\tt lbjava.learn.SparseNetworkLearner}} \label{subsection:SNL}
{\tt SparseNetworkLearner} is a multi-class learner, meaning that it can learn
to distinguish among two or more discrete label values when classifying an
object.  It is not necessary to know which label values are possible when
employing this learner (i.e., it is not necessary for the label producing
classifier specified in a learning classifier expression to be declared with a
value list in its feature return type).  Values that were never observed
during training will never be predicted. \\

This learner creates a new {\tt LinearThresholdUnit} for each label value it
observes and trains each independently to predict {\tt true} when its
associated label value is the correct classification.  When making a
prediction on a new object, it produces the label value corresponding to the
{\tt LinearThresholdUnit} producing the highest score.  The {\tt
LinearThresholdUnit} used may be selected by the programmer, or, if no
specific learner is specified, the default is {\tt SparsePerceptron}. \\

{\tt SparseNetworkLearner} is the default discrete learner; if the programmer
does not include a {\tt with} clause in a learning classifier expression (see
Section \ref{subsection:LCE}) of discrete feature return type, this learner is
invoked with default parameters.

\subsection{{\tt lbjava.learn.NaiveBayes}}
Na\"ive Bayes is a multi-class learner that uses prediction value counts and
feature counts given a particular prediction value to select the most likely
prediction value.  It is not mistake driven, as {\tt LinearThresholdUnit}s
are.  The scores returned by its {\tt scores(Object)} method are directly
interpretable as empirical probabilities.  It also has a smoothing parameter
configurable by the user for dealing with features that were never encountered
during training.

\subsection{{\tt lbjava.learn.StochasticGradientDescent}}
Gradient descent is a batch learning algorithm for function approximation in
which the learner tries to follow the gradient of the error function to the
solution of minimal error.  This implementation is a stochastic approximation
to gradient descent in which the approximated function is assumed to have
linear form. \\

{\tt StochasticGradientDescent} is the default real learner; if the programmer
does not include a {\tt with} clause in a learning classifier expression (see
Section \ref{subsection:LCE}) of real feature return type, this learner is
invoked with default parameters.

\subsection{{\tt lbjava.learn.Normalizer}}
A normalizer is a method that takes a set of scores as input and modifies
those scores so that they obey particular constraints.  Class {\tt Normalizer}
is an abstract class with a single abstract method {\tt
normalize(lbjava.classify.ScoreSet)} (see Section \ref{subsection:scoreSet})
which is implemented by extending classes to define this ``normalization.''
For example:

\begin{list}{}{}
\item[{\tt lbjava.learn.Sigmoid}:] ~\\
This {\tt Normalizer} simply replaces each score $s_i$ in the given {\tt
ScoreSet} with $\frac{1}{1 + e^{s_i}}$.  After normalization, each score will
be greater than 0 and less than 1.

\item[{\tt lbjava.learn.Softmax}:] ~\\
This {\tt Normalizer} replaces each score with the fraction of its exponential
out of the sum of all scores' exponentials.  More precisely, each score $s_i$
is replaced by $\frac{e^{s_i}}{\sum_j e^{s_j}}$.  After normalization, each
score will be positive and they will sum to 1.

\item[{\tt lbjava.learn.IdentityNormalizer}:] ~\\
This {\tt Normalizer} simply returns the same scores it was passed as input.
\end{list}

\subsection{{\tt lbjava.learn.WekaWrapper}}
The {\tt WekaWrapper} class is meant to wrap instances of learners from the
WEKA library of learning
algorithms\footnote{http://www.cs.waikato.ac.nz/ml/weka/}.  The
{\tt lbjava.learn.WekaWrapper} class converts between the internal
representations of LBJava and WEKA on the fly, so that the more extensive set of
algorithms contained within WEKA can be applied to projects written in LBJava.\\

The {\tt WekaWrapper} class extends {\tt lbjava.learn.Learner}, and carries all
of the functionality that can be expected from a learner.  A standard
invocation of {\tt WekaWrapper} could look something like this: \\

{\tt new WekaWrapper(new weka.classifiers.bayes.NaiveBayes())} 

\subsubsection*{Restrictions}
\begin{itemize}

\item
  It is crucial to note that WEKA learning algorithms do not learn online.
  Therefore, whenever the {\tt learn} method of the {\tt WekaWrapper} is
  called, no learning actually takes place.  Rather, the input object is added
  to a collection of examples for the algorithm to learn once the
  {\tt doneLearning()} method is called.

\item
  The {\tt WekaWrapper} only supports features which are either discrete
  without a value list, discrete with a value list, or real.  In WEKA, these
  correspond to {\tt weka.core.Attribute} objects of type {\tt String},
  {\tt Nominal}, and {\tt Numerical}.  In particular, array producing
  classifiers and feature generators may not be used as features for a
  learning classifier learned with this class.  See section
  \ref{section:declarations} for further discussion Classifier Declarations.

\item
  When designing a learning classifier which will use a learning algorithm
  from WEKA, it is important to note that very very few algorithms in the WEKA
  library support {\tt String} attributes.  In LBJava, this means that it will be
  very hard to find a learning algorithm which will learn using a
  {\tt discrete} feature extractor which does not have a value list.  I.e.
  value lists should be provided for discrete feature extracting classifiers
  whenever possible. 

\item
  Feature pre-extraction must be enabled in order to use the {\tt WekaWrapper}
  class.  Feature pre-extraction is enabled by using the {\tt preExtract}
  clause in the {\tt LearningClassifierExpression} (discussed in
  \ref{subsection:LCE}).

\end{itemize}


\section{{\tt lbjava.infer}}

The {\tt lbjava.infer} package contains many classes.  The great majority of
these classes form the internal representation of both propositional and first
order constraint expressions and are used internally by LBJava's inference
infrastructure.  Only the programmer who designs his own inference algorithm
in terms of constraints needs to familiarize himself with these classes.
Detailed descriptions of them are provided in the Javadoc. \\

There are a few classes, however, that are of broader interest.  First, the
{\tt Inference} class is an abstract class from which all inference algorithms
implemented for LBJava are derived.  It is described below along with the
particular algorithms that have already been implemented.  Finally, the {\tt
InferenceManager} class is used internally by the LBJava library when
applications using inference are running.

\subsection{{\tt lbjava.infer.Inference}}

{\tt Inference} is an abstract class from which all inference algorithms are
derived.  Executing an inference generally evaluates all the learning
classifiers involved on the objects they have been applied to in the
constraints, as well as picking new values for their predictions so that the
constraints are satisfied.  An object of this class keeps track of all the
information necessary to perform inference in addition to the information
produced by it.  Once that inference has been performed, constrained
classifiers access the results through this class's interface to determine
what their constrained predictions are.  This is done through the {\tt
valueOf(lbjava.learn.Learner, Object)} method described below.

\begin{list}{}{}
\item[{\tt String valueOf(lbjava.learn.Learner, Object)}:] ~\\
The arguments to this method are objects representing a learning classifier
and an object involved in the inference.  Calling this method causes the
inference algorithm to run, if it has not been run before.  This method then
returns the new prediction corresponding to the given learner and object after
constraints have been resolved.
\end{list}

\subsection{{\tt lbjava.infer.GLPK}} \label{subsection:GLPK}

This inference algorithm, which may be named in the {\tt with} clause of the
LBJava {\tt inference} syntax, uses Integer Linear Programming (ILP) to maximize
the expected number of correct predictions while respecting the constraints.
Upon receiving the constraints represented as First Order Logic (FOL)
formulas, this implementation first translates those formulas to a
propositional representation.  The resulting propositional expression is then
translated to a set of linear inequalities by recursively translating
subexpressions into sets of linear inequalities that bound newly created
variables to take their place. \\

The number of linear inequalities and extra variables generated is linear in
the depth of the tree formed by the propositional representation of the
constraints.  This tree is not binary; instead, nodes representing operators
that are associative and commutative such as conjunction and disjunction have
multiple children and are not allowed to have children representing the same
operator (i.e., when they do, they are collapsed into the parent node).  So
both the number of linear inequalities and the number of extra variables
created will be relatively low.  However, the performance of any ILP algorithm
is very sensitive to both these numbers, since ILP is NP-hard.  On a 3 Ghz
machine, the programmer will still do well to keep both these numbers under
20,000 for any given instance of the inference problem. \\

The resulting ILP problem is then solved by the GNU Linear Programming Kit
(GLPK), a linear programming library written in
C.\footnote{http://www.gnu.org/software/glpk/}  This software must be
downloaded and installed separately before installing LBJava, or the {\tt GLPK}
inference algorithm will be disabled.  If LBJava has already been installed, it
must be reconfigured and reinstalled (see Chapter \ref{section:installation})
after installing GLPK.

\section{{\tt lbjava.parse}}

This package contains the very simple {\tt Parser} interface, implementers of
which are used in conjunction with learning classifier expressions in an LBJava
source file when off-line training is desired (see Section
\ref{subsection:LCE}).  It also contains some general purpose internal
representations which may be of interest to a programmer who has not yet
written the internal representations or parsers for the application.

\subsection{{\tt lbjava.parse.Parser}} \label{subsection:parser}
The LBJava compiler is capable of automatically training a learning classifier
given training data, so long as that training data comes in the form of
objects ready to be passed to the learner's {\tt learn(Object)} method.  Any
class that implements the {\tt Parser} interface can be utilized by the
compiler to provide those training objects.  This interface simply consists of
a single method for returning another object:

\begin{list}{}{}
\item[{\tt Object next()}:] ~\\
This is the only method that an implementing class needs to define.  It
returns the next training {\tt Object} until no more are available, at which
point it returns {\tt null}.
\end{list}

\subsection{{\tt lbjava.parse.LineByLine}} \label{subsection:lineByLine}
This abstract class extends {\tt Parser} but does not implement the {\tt
next()} method.  It does, however, define a constructor that opens the file
with the specified name and a readLine() method that fetches the next line of
text from that file.  Exceptions (as may result from not being able to open or
read from the file) are automatically handled by printing an error message and
exiting the application.

\subsection{{\tt lbjava.parse.ChildrenFromVectors}} \label{subsection:CFV}
This parser calls a user specified, {\tt LinkedVector} (see Section
\ref{subsection:linkedVector}) returning {\tt Parser} internally and returns
the {\tt LinkedChild}ren (see Section \ref{subsection:linkedChild}) of that
vector one at a time through its {\tt next()} method.  One notable {\tt
LinkedVector} returning {\tt Parser} is {\tt lbjava.nlp.WordSplitter} discussed
in Section \ref{subsection:parsers}.

\subsection{{\tt lbjava.parse.FeatureVectorParser}} \label{subsection:FVP}
This parser is used internally by the LBJava compiler (and may be used by the
programmer as well) to continue training the learning classifier after the
first round of training without incurring the cost of feature extraction.  See
Section \ref{subsection:LCE} for more information on LBJava's behavior when the
programmer specifies multiple training rounds.  That section describes how
lexicon and example files are produced, and these files become the input to
{\tt FeatureVectorParser}. \\

The objects produced by {\tt FeatureVectorParser} will be {\tt
FeatureVector}s, which are not normally the input to any classifier, including
the learning classifier we'd like to continue training.  So, the programmer
must first replace the learning classifier's feature extractor with a {\tt
FeatureVectorReturner} and its labeler with a {\tt LabelVectorReturner} (see
Section \ref{subsection:vectorReturners}) before calling {\tt learn(Object)}.
After the new training objects have been exhausted, the original feature
extractor and labeler must be restored before finally calling {\tt save()}. \\

For example, if a learning classifier named {\tt MyTagger} has been trained
for multiple rounds by the LBJava compiler, the lexicon and example file will be
created with the names {\tt MyTagger.lex} and {\tt MyTagger.ex} respectively.
Then the following code in an application will continue training the
classifier for an additional round: \\

\vspace{-.25cm}
{\tt MyTagger tagger = new MyTagger();}

{\tt Classifier extractor = tagger.getExtractor();}

{\tt tagger.setExtractor(new FeatureVectorReturner());}

{\tt Classifier labeler = tagger.getLabeler();}

{\tt tagger.setLabeler(new LabelVectorReturner());}

{\tt FeatureVectorParser parser =}

\hspace{.4cm} {\tt new FeatureVectorParser("MyTagger.ex", "MyTagger.lex");}

{\tt for (Object vector = parser.next(); vector != null;
          vector = parser.next())}

\hspace{.4cm} {\tt tagger.learn(vector);}

{\tt tagger.setExtractor(extractor);}

{\tt tagger.setLabeler(labeler);}

{\tt tagger.save();}

\subsection{{\tt lbjava.parse.LinkedChild}} \label{subsection:linkedChild}
Together with {\tt LinkedVector} discussed next, these two classes form the
basis for a simple, general purpose internal representation for raw data.
{\tt LinkedChild} is an abstract class containing pointers to two other {\tt
LinkedChild}ren, the ``previous'' one and the ``next'' one.  It may also store
a pointer to its parent, which is a {\tt LinkedVector}.  Constructors that set
up all these links are also provided, simplifying the implementation of the
parser.

\subsection{{\tt lbjava.parse.LinkedVector}} \label{subsection:linkedVector}
A {\tt LinkedVector} contains any number of {\tt LinkedChild}ren and provides
random access to them in addition to the serial access provided by their
links.  It also provides methods for insertion and removal of new children.  A
{\tt LinkedVector} is itself also a {\tt LinkedChild}, so that hierarchies are
easy to construct when sub-classing these two classes.

\section{{\tt lbjava.nlp}}

The programmer of Natural Language Processing (NLP) applications may find the
internal representations and parsing algorithms implemented in this package
useful.  There are representations of words, sentences, and documents, as well
as parsers of some common file formats and algorithms for word and sentence
segmentation.

\subsection{Internal Representations} \label{subsection:IR}

These classes may be used to represent the elements of a natural language
document.

\begin{list}{}{}
\item[{\tt lbjava.nlp.Word}:] ~\\
This simple representation of a word extends the {\tt LinkedChild} class (see
Section \ref{subsection:linkedChild}) and has space for its spelling and part
of speech tag.

\item[{\tt lbjava.nlp.Sentence}:] ~\\
Objects of the {\tt Sentence} class store only the full text of the sentence
in a single {\tt String}.  However, a method is provided to heuristically
split that text into {\tt Word} objects contained in a {\tt LinkedVector}.

\item[{\tt lbjava.nlp.NLDocument}:] ~\\
Extended from {\tt LinkedVector}, this class has a constructor that takes the
full text of a document as input.  Using the methods in {\tt Sentence} and
{\tt SentenceSplitter}, it creates a hierarchical representation of a natural
language document in which {\tt Word}s are contained in {\tt LinkedVector}s
representing sentences which are contained in this {\tt LinkedVector}.

\item[{\tt lbjava.nlp.POS}:] ~\\
This class may be used to represent a part of speech, but it used more
frequently to simply retrieve information about the various parts of speech
made standard by the Penn Treebank project \cite{marcus94building}.
\end{list}

\subsection{Parsers} \label{subsection:parsers}

The classes listed in this section are all derived from class {\tt LineByLine}
(see Section \ref{subsection:lineByLine}).  They all contain (at least) a
constructor that takes a single {\tt String} representing the name of a file
as input.  The objects they return are retrieved through the overridden {\tt
next()} method.

\begin{list}{}{}
\item[{\tt lbjava.nlp.SentenceSplitter}:] ~\\
Use this {\tt Parser} to separate sentences out from plain text.  The class
provides two constructors, one for splitting sentences out of a plain text
file, and the other for splitting sentences out of plain text already stored
in memory in a {\tt String[]}.  The user can then retrieve {\tt
Sentence}s one at a time with the {\tt next()} method, or all at once with
the {\tt splitAll()} method.  The returned {\tt Sentence}s' start and end
fields represent offsets into the text they were extracted from.  Every
character in between those two offsets inclusive, including extra spaces,
newlines, etc., is included in the {\tt Sentence} as it appeared in the
paragraph.\footnote{If the constructor taking a {\tt String[]} as an
argument is used, newline characters are inserted into the returned sentences
to indicate transitions from one element of the array to the next.}

\item[{\tt lbjava.nlp.WordSplitter}:] ~\\
This parser takes the plain, unannotated {\tt Sentence}s (see Section
\ref{subsection:IR}) returned by another parser (e.g., {\tt SentenceSplitter})
and splits them into Word objects.  Entire sentences now represented as {\tt
LinkedVector}s (see Section \ref{subsection:linkedVector}) are then returned
one at a time by calls to the {\tt next()} method.

\item[{\tt lbjava.nlp.ColumnFormat}:] ~\\
This parser returns a {\tt String[]} representing the rows of a file in column
format.  The input file is assumed to contain fields of non-whitespace
characters separated by any amount of whitespace, one line of which is
commonly used to represent a word in a corpus.  This parser breaks a given
line into one {\tt String} per field, omitting all of the whitespace.  A
common usage of this class will be in extending it to create a new {\tt
Parser} that calls {\tt super.next()} and creates a more interesting internal
representation with the results.

\item[{\tt lbjava.nlp.POSBracketToVector}:] ~\\
Use this parser to return {\tt LinkedVector} objects representing sentences
given file names of POS bracket form files to parse.  These files are expected
to have one sentence per line, and the format of each line is as follows: \\

\vspace{-.25cm}
{\tt (pos$_1$ spelling$_1$) (pos$_2$ spelling$_2$) ... (pos$_n$ spelling$_n$)}
\\
\vspace{-.25cm}

\noindent
It is also expected that there will be exactly one space between a part of
speech and the corresponding spelling and between a closing parenthesis and an
opening parenthesis.

\end{list}

