
\chapter{The LBJava Language} \label{chapter:language}

Now that we have defined the building blocks of classifier computation, we
next describe LBJava's syntax and semantics for programming with these building
blocks. \\

Like a Java source file, an LBJava source file begins with an optional package
declaration and an optional list of import declarations.  Next follow the
definitions of classifiers, constraints, and inferences.  Each will be
translated by the LBJava compiler into a Java class of the same name.  If the
package declaration is present, those Java classes will all become members of
that package.  Import declarations perform the same function in an LBJava source
file as in a Java source file.

\section{Classifiers}

In LBJava, a classifier can be defined with Java code or composed from the
definitions of other classifiers using special operators.  As such, the syntax
of classifier specification allows the programmer to treat classifiers as
expressions and assign them to names.  This section defines the syntax of
classifier specification more precisely, including the syntax of classifiers
learned from data.  It also details the behavior of the LBJava compiler when
classifiers are specified in terms of training data and when changes are made
to an LBJava source file.

\subsection{Classifier Declarations} \label{section:declarations}
Classifier declarations are used to name classifier expressions (discussed in
Section \ref{section:classifierExpressions}).  The syntax of a classifier
declaration has the following form: \\

\vspace{-.25cm}
{\tt \emph{feature-type name}(\emph{type name}) \\
  \mbox{\hspace{1cm}}$[$cached $|$ cachedin \emph{field-access}$]$ <- \\
  \mbox{\hspace{1cm}}\emph{classifier-expression}} \\
\vspace{-.25cm}

\noindent
A classifier declaration names a classifier and specifies its input and output
types in its header, which is similar to a Java method header.
It ends with a left arrow indicating assignment and a classifier expression
which is assigned to the named classifier. \\

The optional {\tt cached} and {\tt cachedin} keywords are used to indicate
that the result of this classifier's computation will be cached in association
with the input object.  The {\tt cachedin} keyword instructs the classifier to
cache its output in the specified field of the input object.  For example, if
the parameter of the classifier is specified as {\tt Word w}, then
{\tt w.partOfSpeech} may appear as the {\tt\emph{field-access}}.  The
{\tt cached} keyword instructs the classifier to store the output values it
computes in a hash table.  As such, the implementations of the
{\tt hashCode()} and {\tt equals(Object)} methods in the input type's class
play an important role in the behavior of a {\tt cached} classifier.  If two
input objects return the same value from their {\tt hashCode()} methods and
are equivalent according to {\tt equals(Object)}, they will receive the same
classification from this classifier.\footnote{Because this type of classifier
caching is implemented with a {\tt java.util.WeakHashMap}, it is possible for
this statement to be violated if the two objects are not alive in the heap
simultaneously.  For more information, see the Java API javadoc.} \\

A cached classifier (using either type of caching) will first check the
specified appropriate location to see if a value has already been computed for
the given input object.  If it has, it is simply returned.  Otherwise, the
classifier computes the value and stores it in the location before returning
it.  A discrete classifier will store its value as a {\tt String}.  When
caching in a field, it will assume that {\tt null} represents the absence of a
computed value.  A real classifier will store its value as a {\tt double}.
When caching in a field, it will assume that {\tt Double.NaN} represents the
absence of a computed value.  Array returning classifiers will store a value
as an array of the appropriate type and assume that {\tt null} represents the
absence of a computed value.  Generators may not be cached with either
keyword.  Last but not least, learning classifiers cached with either keyword
will not load their (potentially large) internal representations from disk
until necessary (i.e., until an object is encountered whose cache location is
not filled in).  See Section \ref{subsection:LCE} for more information about
learning classifiers. \\

Semantically, every named classifier is a static method.  In an LBJava source
file, references to classifiers are manipulated and passed to other syntactic
constructs, similarly to a functional programming language.  The LBJava compiler
implements this behavior by storing a classifier's definition in a static
method of a Java class of the same name and providing access to that method
through objects of that class.  As we will see, learning classifiers are
capable of modifying their definition, and by the semantics of classifier
declarations, these modifications are local to the currently executing
process, but not to any particular object.  In other words, when the
application continues to train a learning classifier on-line, the changes are
immediately visible through every object of the classifier's class. \\

\begin{figure}
\begin{center}
\begin{algorithm}
    {\tt discrete\{false, true\} highRisk(Patient p) <-}
\\  {\tt \{ return p.historyOfCancer() \&\& p.isSmoker(); \}}
\\
\\  {\tt discrete prefix(Word w) cachedin w.prefix <-}
\\  {\tt \{} \+
\\  {\tt  if (w.spelling > 5) {\tt return} w.spelling.substring(0, 3);}
\\  {\tt  return w.spelling;} \-
\\  {\tt \}}
\\
\\  {\tt real[] dimensions(Cube c) <-}
\\  {\tt \{} \+
\\  {\tt  sense c.length;}
\\  {\tt  sense c.width;}
\\  {\tt  sense c.height;} \-
\\  {\tt \}}
\\
\\  {\tt discrete bigram(Word w) <- spellingTarget \&\& spellingOneAfter}
\end{algorithm}
\end{center}
\caption{Classifier declarations declaring hard-coded classifiers.}
\label{figure:classifierDeclarations}
\end{figure}

Figure \ref{figure:classifierDeclarations} gives several examples of
classifier declarations.  These examples illustrate some key principles LBJava.
First, the features produced by a classifier are either discrete or real.  If
a feature is discrete, the set of allowable values may optionally be
specified, contained in curly braces.  Any literal values including {\tt
int}s, {\tt String}s, and {\tt boolean}s may be used in this
set\footnote{Internally, they'll all be converted to {\tt String}s.}. \\

Next, every classifier takes exactly one object as input and returns one or
more features as output.  The input object will most commonly be an object
from the programmer-designed, object-oriented internal representation of the
application domain's data.  When the classifier has made its
classification(s), it returns one or more features representing those
decisions.  A complete list of feature return types follows:

\begin{itemize}
\item {\tt discrete}
\item {\tt discrete\{ \emph{value-list} \}}
\item {\tt real}
\item {\tt discrete[]}
\item {\tt discrete\{ \emph{value-list} \}[]}
\item {\tt real[]}
\item {\tt discrete\%}
\item {\tt discrete\{ \emph{value-list} \}\%}
\item {\tt real\%}
\item {\tt mixed\%}
\end{itemize}

Feature return types ending with square brackets indicate that an array of
features is produced by this classifier.  The user can expect the feature at a
given index of the array to be the same feature with a differing value each
time the classifier is called on a different input object.  Feature return
types ending with a percent sign indicate that this classifier is a
\emph{feature generator}.  A feature generator may return zero or more
features in any order when it is called, and there is no guarantee that the
same features will be produced when called on different input objects.
Finally, the {\tt mixed\%} feature return type indicates that the classifier
is a generator of both discrete and real features. \\

As illustrated by the fourth classifier in Figure
\ref{figure:classifierDeclarations}, a classifier may be composed from other
classifiers with classifier operators.  The names {\tt spellingTarget} and
{\tt spellingOneAfter} here refer to classifiers that were either defined
elsewhere in the same source file or that were imported from some other
package.  In this case, the classifier {\tt bigram} will return a discrete
feature whose value is the conjunction of the values of the features produced
by {\tt spellingTarget} and {\tt spellingOneAfter}. \\

All of the classifiers in Figure \ref{figure:classifierDeclarations} are
examples of explicitly coded classifiers.  The first three use Java method
bodies to compute the values of the returned features.  In each of these
cases, the names of the returned features are known directly from the
classifier declaration's header, so their values are all that is left to
compute.  In the first two examples, the header indicates that a single
feature will be returned.  Thus, the familiar {\tt return} statement is used
to indicate the feature's value.  In the third example, the square brackets in
the header indicate that this classifier produces an array of features.
Return statements are disallowed in this context since we must return
multiple values.  Instead, the {\tt sense} statement is used whenever the
next feature's value is computed. \\

For our final example, we will demonstrate the specification of a learning
classifier.  The {\tt learnMe} learning classifier in Figure
\ref{figure:learnMe} is supervised by a classifier named
{\tt labelingClassifier} whose features will be interpreted as labels of an
input training object.  Next, after the {\tt using} clause appears a comma
separated list of classifier names.  These classifiers perform feature
extraction.  The optional {\tt from} clause designates a parser used to
provide training objects to {\tt learnMe} at compile-time.  Finally, the
optional {\tt with} clause designates a particular learner for {\tt learnMe}
to utilize.

\begin{figure}
\begin{center}
\begin{algorithm}
    {\tt discrete learnMe(InputObject o) <-}
\\  {\tt learn} \+
\\  {\tt  labelingClassifier}
\\  {\tt  using c1, c2, c3}
\\  {\tt  from new UserDefinedParser(data)}
\\  {\tt  with new PreDefinedLearner(parameters)} \-
\\  {\tt end}
\end{algorithm}
\end{center}
\caption{Learning classifier specification}
\label{figure:learnMe}
\end{figure}

\subsection{Classifier Expressions} \label{section:classifierExpressions}

As was alluded to above, the right hand side of a classifier declaration is
actually a single classifier expression.  A classifier expression is one of
the following syntactic constructs:

\begin{itemize}
\item a classifier name
\item a method body (i.e., a list of Java statements in between curly braces)
\item a classifier cast expression
\item a conjunction of two classifier expressions
\item a comma separated list of classifier expressions
\item a learning classifier expression
\item an inference invocation expression
\end{itemize}

\noindent
We have already explored examples of almost all of these.  More precise
definitions of each follow.

\subsubsection{Classifier Names} \label{subsection:classifierNames}
The name of a classifier defined either externally or in the same source file
may appear wherever a classifier expression is expected.  If the named
classifier's declaration is found in the same source file, it may occur
anywhere in that source file (in other words, a classifier need not be defined
before it is used).  If the named classifier has an external declaration it
must either be fully qualified (e.g., {\tt myPackage.myClassifier}) or it must
be imported by an import declaration at the top of the source file.  The class
file or Java source file containing the implementation of an imported
classifier must exist prior to running the LBJava compiler on the source file
that imports it.

\subsubsection{Method Bodies}
A method body is a list of Java statements enclosed in curly braces explicitly
implementing a classifier.  When the classifier implemented by the method body
returns a single feature, the {\tt return} statement is used to provide that
feature's value.  If the feature return type is {\tt real}, then the {\tt
return} statement's expression must evaluate to a {\tt double}.  Otherwise, it
can evaluate to anything - even an object - and the resulting value will be
converted to a {\tt String}.  Each method body takes its argument and feature
return type from the header of the classifier declaration it is contained in
(except when in the presence of a classifier cast expression, discussed in
Section \ref{subsection:castExpressions}).  For more information on method
bodies in LBJava, see Section \ref{section:methodBodies}.

\subsubsection{Classifier Cast Expressions} \label{subsection:castExpressions}
When the programmer wishes for a classifier sub-expression on the right hand
side of a classifier declaration to be implemented with a feature return type
differing from that defined in the header, a classifier cast expression is the
solution.  For example, the following classifier declaration exhibits a
learning classifier (see Section \ref{subsection:LCE}) with a real valued
feature return type.  One of the classifiers it uses as a feature extractor is
hard-coded on the fly, but it returns a discrete feature.  A classifier cast
expression is employed to achieve the desired affect.

\begin{center}
\begin{algorithm}
    {\tt real dummyClassifier(InputObject o) <-}
\\  {\tt learn} \+
\\  {\tt  labeler}
\\  {\tt  using c1, c2, (discrete) \{ return o.value == 4; \}} \-
\\  {\tt end}
\end{algorithm}
\end{center}

\noindent
Of course, we can see that the hard-coded classifier defined on the fly in
this example returns a discrete ({\tt boolean}) value.  Without the cast in
front of this method body, the LBJava compiler would have assumed it to have a
real valued feature return type, and an error would have been produced. \\

When a classifier cast expression is applied to a classifier expression that
contains other classifier expressions, the cast propagates down to those
classifier expressions recursively as well.

\subsubsection{Conjunctions}
\begin{table}
\begin{center}
\begin{tabular}{|l|l|l|}
\hline
{\bf Argument Type} & {\bf Argument Type} & {\bf Result Type} \\ \hline
{\tt discrete}      & {\tt discrete}      & {\tt discrete}    \\ \hline
{\tt discrete}      & {\tt real}          & {\tt real\%}      \\ \hline
{\tt discrete}      & {\tt discrete[]}    & {\tt discrete[]}  \\ \hline
{\tt discrete}      & {\tt real[]}        & {\tt real\%}      \\ \hline
{\tt discrete}      & {\tt discrete\%}    & {\tt discrete\%}  \\ \hline
{\tt discrete}      & {\tt real\%}        & {\tt real\%}      \\ \hline
{\tt discrete}      & {\tt mixed\%}       & {\tt mixed\%}     \\ \hline
{\tt real}          & {\tt real}          & {\tt real}        \\ \hline
{\tt real}          & {\tt discrete[]}    & {\tt real\%}      \\ \hline
{\tt real}          & {\tt real[]}        & {\tt real[]}      \\ \hline
{\tt real}          & {\tt discrete\%}    & {\tt real\%}      \\ \hline
{\tt real}          & {\tt real\%}        & {\tt real\%}      \\ \hline
{\tt real}          & {\tt mixed\%}       & {\tt mixed\%}     \\ \hline
{\tt discrete[]}    & {\tt discrete[]}    & {\tt discrete[]}  \\ \hline
{\tt discrete[]}    & {\tt real[]}        & {\tt real\%}      \\ \hline
{\tt discrete[]}    & {\tt discrete\%}    & {\tt discrete\%}  \\ \hline
{\tt discrete[]}    & {\tt real\%}        & {\tt real\%}      \\ \hline
{\tt discrete[]}    & {\tt mixed\%}       & {\tt mixed\%}     \\ \hline
{\tt real[]}        & {\tt real[]}        & {\tt real[]}      \\ \hline
{\tt real[]}        & {\tt discrete\%}    & {\tt real\%}      \\ \hline
{\tt real[]}        & {\tt real\%}        & {\tt real\%}      \\ \hline
{\tt real[]}        & {\tt mixed\%}       & {\tt mixed\%}     \\ \hline
{\tt discrete\%}    & {\tt discrete\%}    & {\tt discrete\%}  \\ \hline
{\tt discrete\%}    & {\tt real\%}        & {\tt real\%}      \\ \hline
{\tt discrete\%}    & {\tt mixed\%}       & {\tt mixed\%}     \\ \hline
{\tt real\%}        & {\tt real\%}        & {\tt real\%}      \\ \hline
{\tt real\%}        & {\tt mixed\%}       & {\tt mixed\%}     \\ \hline
{\tt mixed\%}       & {\tt mixed\%}       & {\tt mixed\%}     \\ \hline
\end{tabular}
\caption{Conjunction feature return types given particular argument classifier
feature return types.}
\label{table:conjunctionReturnTypes}
\end{center}
\end{table}

A conjunction is written with the double ampersand operator ({\tt \&\&}) in
between two classifier expressions (see Figure
\ref{figure:classifierDeclarations} for an example).  The conjunction of two
classifiers results in a new classifier that combines the values of the
features returned by its argument classifiers.  The nature of the combination
depends on the feature return types of the argument classifiers.  Table
\ref{table:conjunctionReturnTypes} enumerates all possibilities and gives the
feature return type of the resulting conjunctive classifier. \\

In general, the following rules apply.  Two discrete features are combined
simply through concatenation of their values.  One discrete and one real
feature are combined by creating a new real valued feature whose name is a
function of the discrete feature's value and whose value is equal to the real
feature's value.  When two real features are combined, their values are
multiplied. \\

The conjunction of two classifiers that return a single feature is a
classifier returning a single feature.  When a classifier returning an array
is conjuncted with a either a single feature classifier or a classifier
returning an array, the result is an array classifier whose returned array
will contain the combinations of every pairing of features from the two
argument classifiers.  Finally, the conjunction of a feature generator with
any other classifier will result in a feature generator producing features
representing the combination of every pairing of features from the two
argument classifiers.

\subsubsection{Composite Generators}
``Composite generator'' is LBJava terminology for a comma separated list of
classifier expressions.  When classifier expressions are listed separated by
commas, the result is a feature generator that simply returns all the features
returned by each classifier in the list.

\subsubsection{Learning Classifier Expressions} \label{subsection:LCE}
Learning classifier expressions have the following syntax: \\

\vspace{-.25cm}
\begin{tabular}{ll}
{\tt learn $[$\emph{classifier-expression}$]$} &
{\tt // Labeler} \\
\hspace{.4cm}{\tt using \emph{classifier-expression}} &
{\tt // Feature extractors} \\
\hspace{.4cm}{\tt $[$from \emph{instance-creation-expression}
                          $[$\emph{int}$]]$} &
{\tt // Parser} \\
\hspace{.4cm}{\tt $[$with \emph{instance-creation-expression}$]$} &
{\tt // Learning algorithm} \\
\hspace{.4cm}{\tt $[$evaluate \emph{Java-expression}$]$} &
{\tt // Alternate eval method} \\
\hspace{.4cm}{\tt $[$cval $[$\emph{int}$]$ \emph{split-strategy}$]$} &
{\tt // K-Fold Cross Validation} \\
\hspace{.6cm}{\tt $[$alpha \emph{double}$]$} &
{\tt // Confidence Parameter} \\
\hspace{.6cm}{\tt $[$testingMetric} & \\
\hspace{1cm}{\tt \emph{instance-creation-expression}$]]$} &
{\tt // Testing Function} \\
\hspace{.4cm}{\tt $[$preExtract \emph{boolean}$]$} &
{\tt // Feature Pre-Extraction} \\
\hspace{.4cm}{\tt $[$progressOutput \emph{int}$]$} &
{\tt // Progress Output Frequency} \\
{\tt end} &
\end{tabular} \\

\noindent
The first classifier expression represents a classifier that will provide
label features for a supervised learning algorithm.  It need not appear when
the learning algorithm is unsupervised.\footnote{Keep in mind, however, that
LBJava's library currently lacks unsupervised learning algorithm
implementations.}  The classifier expression in the {\tt using} clause does
all the feature extraction on each object, during both training and
evaluation.  It will often be a composite generator. \\

The instance creation expression in the {\tt from} clause should create an
object of a class that implements the {\tt lbjava.parser.Parser} interface in
the library (see Section \ref{subsection:parser}).  This clause is optional.
If it appears, the LBJava compiler will automatically perform training on the
learner represented by this learning classifier expression at compile-time.
Whether it appears or not, the programmer may continue training the learner
on-line in the application via methods defined in {\tt lbjava.learn.Learner} in
the library (see Section \ref{subsection:learner}). \\

When the {\tt from} clause appears, the LBJava compiler retrieves objects from
the specified parser until it finally returns {\tt null}.  One at a time, the
feature extraction classifier is applied to each object, and the results are
sent to the learning algorithm for processing.  However, many learning
algorithms perform much better after being given multiple opportunities to
learn from each training object.  This is the motivation for the integer
addendum to this clause.  The integer specifies a number of \emph{rounds}, or
the number of passes over the training data to be performed by the classifier
during training.  \\

The instance creation expression in the {\tt with} clause should create an
object of a class derived from the {\tt lbjava.learn.Learner} class in the
library.  This clause is also optional.  If it appears, the generated Java
class implementing this learning classifier will be derived from the class
named in the {\tt with} clause.  Otherwise, the default learner for the
declared return type of this learning classifier will be substituted with
default parameter settings. \\

The {\tt evaluate} clause is used to specify an alternate method for
evaluation of the learned classifier.  For example, the {\tt
SparseNetworkLearner} learner is a multi-class learner that, during
evaluation, predicts the label for which it computes the highest score.
However, it also provides the {\tt valueOf(Object, java.util.Collection)}
method which restricts the prediction to one of the labels in the specified
collection.  In the application, it's easy enough to call this method in place
of {\tt discreteValue(Object)} (discussed in Section
\ref{subsection:classifier}), but when this classifier is invoked elsewhere in
an LBJava source file, it translates to an invocation of {\tt
discreteValue(Object)}.  The {\tt evaluate} clause (e.g., {\tt evaluate
valueOf(o, MyClass.getCollection())}) changes the behavior of {\tt
discreteValue(Object)} (or {\tt realValue(Object)} as appropriate) so that it
uses the specified {\tt\emph{Java-expression}} to produce the prediction.
Note that {\tt\emph{Java-expression}} will be used only during the evaluation
and not the training of the learner specifying the {\tt evaluate} clause. \\

The {\tt cval} clause enables LBJ's built-in $K$-fold cross validation system.
$K$-fold cross validation is a statistical technique for assessing the
performance of a learned classifier by partitioning the user's set of training
data into $K$ subsets such that a single subset is held aside for testing
while the others are used for training. LBJava automates this process in order to
alleviate the need for the user to perform his own testing methodologies. The
optional {\tt split-strategy} argument to the {\tt cval} clause can be used to
specify the method with which LBJava will split the data set into subsets
(folds). If the {\tt split-strategy} argument is not provided, the default
value taken is {\tt sequential}. The user may choose from the following four
split strategies:

\begin{itemize}
\item {\tt sequential} -
  The {\tt sequential} split strategy attempts to partition the set of
  examples into $K$ equally sized subsets based on the order in which they are
  returned from the user's parser. Given that there are $T$ examples in the
  data set, the first $T/K$ examples encountered are considered to be the
  first subset, while the examples between the $(T/K+1)$'th example and the
  $(2T/K)$'th example are considered to be the second subset, and so on. \\
  i.e. [ --- 1 --- $|$ --- 2 --- $| $ ... $|$ --- $K$ --- ]

\item {\tt kth} -
  The {\tt kth} split strategy also attempts to partition the set of examples
  in to $K$ equally sized subsets with a round-robin style assignement scheme.
  The $x$'th example encountered is assigned to the $(x\%K)$'th subset. \\
  i.e. [ 1 2 3 4 ... $K$ 1 2 3 4 ... $K$ ... ]

\item {\tt random} -
  The {\tt random} split strategy begins with the assignment given by the
  {\tt kth} split strategy, and simply mixes the subset assignments. This
  ensures that the subsets produced are as equally sized as possible.

\item {\tt manual} -
  The user may write their parser so that it returns the unique instance of
  the {\tt lbjava.parse.FoldSeparator} class (see the {\tt separator} field)
  wherever a fold boundary is desired.  Each time this object appears, it
  represents a partition between two folds.  Thus, if the $k$-fold cross
  validation is desired, it should appear $k-1$ times.  The integer provided
  after the {\tt cval} keyword is ignored and may be omitted in this case.

\end{itemize}

The {\tt testingMetric} and {\tt alpha} clauses are sub-clauses of {\tt cval},
and, consequently, have no effect when the {\tt cval} clause is not present.  The
{\tt testingMetric} clause gives the user the opportunity to provide a custom
testing methodology. The object provided to the {\tt testingMetric} clause
must implement the {\tt lbjava.learn.TestingMetric} interface.  If this clause
is not provided, then it will default to the {\tt lbjava.learn.Accuracy} metric,
which simply returns the ratio of correct predictions made by the classifier
on the testing fold to the total number of examples contained within said
fold.\\

LBJ's cross validation system provides a confidence interval according to the
measurements made by the testing function. With the {\tt alpha} clause, the
user may define the width of this confidence interval. The double-precision
argument provided to the {\tt alpha} clause causes LBJava to calculate a
$(1-a)\%$ confidence interval. For example, ``{\tt alpha .07}" causes LBJava to
print a $93\%$ confidence interval, according to the testing measurements
made. If this clause is not provided, the default value taken is .05, resulting 
in a 95\% confidence interval.\\

The {\tt preExtract} clause enables or disables the pre-extraction of features
from examples. When the argument to this clause is  {\tt true}, feature
extraction is only performed once, the results of which are recorded in two
files.  First, an array of all {\tt Feature} objects (see Section
\ref{subsection:feature}) observed during training is serialized and written
to a file whose name is the same as the learning classifier's and whose
extension is {\tt .lex}.  This file is referred to as the \emph{lexicon}.
Second, the integer indexes, as they are found in this array, of all features
corresponding to each training object are written to a file whose name is the
same as the learning classifier's and whose extension is {\tt .ex}.  This file
is referred to as the \emph{example file}.  It is re-read from disk during
each training round during both cross validation and final training, saving
time when feature extraction is expensive, which is often the case. \\
If this clause is not provided, the default value taken is {\tt false}.\\

The {\tt progressOutput} clause defines how often to produce an output message
during training. The argument to this clause is an integer which represents
the number of examples to process between progress messages. This variable 
can also be set via a command line parameter using the {\tt -t} option. If a value 
is provided in both places, the one defined here in the Learning Classifier Expression 
takes precedence. If no value is provided, then the default value taken is 0, 
causing progress messages to be given only at the beginning and end of each 
training pass.  \\

When the LBJava compiler finally processes a learning classifier expression, it
generates not only a Java source file implementing the classifier, but also a
file containing the results of the computations done during training.  This
file will have the same name as the classifier but with a {\tt .lc} extension
(``lc'' stands for ``learning classifier'').  The directory in which this file
and also the lexicon and example files mentioned earlier are written depends
on the appearance of certain command line parameters discussed in Section
\ref{section:commandLine}.

\subsubsection{Inference Invocations}
\label{subsubsection:inferenceInvocations}
Inference is the process through which classifiers constrained in terms of
each other reconcile their outputs.  More information on the specification of
constraints and inference procedures can be found in Sections
\ref{section:constraints} and \ref{section:inference} respectively.  In LBJava,
the application of an inference to a learning classifier participating in that
inference results in a new classifier whose output respects the inference's
constraints.  Inferences are applied to learning classifiers via the inference
invocation, which looks just like a method invocation with a single argument.
\\

For example, assume that {\tt LocalChunkType} is the name of a discrete
learning classifier involved in an inference procedure named {\tt
ChunkInference}.  Then a new version of {\tt LocalChunkType} that respects the
constraints of the inference may be named as follows: \\

\vspace{-.25cm}
{\tt discrete ChunkType(Chunk c) <- ChunkInference(LocalChunkType)}

\subsection{Method Bodies} \label{section:methodBodies}

Depending on the feature return type, the programmer will have differing needs
when designing a method body.  If the feature return type is either {\tt
discrete} or {\tt real}, then only the value of the single feature is returned
through straight forward use of the {\tt return} statement.  Otherwise,
another mechanism will be required to return multiple feature values in the
case of an array return type, or multiple feature names and values in the case
of a feature generator.  That mechanism is the {\tt sense} statement,
described in Section \ref{subsection:senseStatement}. \\

When a classifier's only purpose is to provide information to a {\tt Learner}
(see Section \ref{subsection:learner}), the {\tt Feature} data type (see
Section \ref{subsection:feature}) is the most appropriate mode of
communication.  However, in any LBJava source file, the programmer will
inevitably design one or more classifiers intended to provide information
within the programmer's own code, either in the application or in other
classifier method bodies.  In these situations, the features' values (and not
their names) are the data of interest.  Section
\ref{subsection:invokingClassifiers} discusses a special semantics for
classifier invocation.

\subsubsection{The Sense Statement} \label{subsection:senseStatement}
The {\tt sense} statement is used to indicate that the name and/or value of a
feature has been detected when computing an array of features or a feature
generator.  In these contexts, any number of features may be sensed, and they
are returned in the order in which they were sensed. \\

The syntax of a {\tt sense} statement in an array returning classifier is
simply \\

\vspace{-.25cm}
{\tt sense \emph{expression};} \\
\vspace{-.25cm}

\noindent
The expression is interpreted as the value of the next feature sensed.  No
name need be supplied, as the feature's name is simply the concatenation of
the classifier's name with the index this feature will take in the array.
This expression must evaluate to a {\tt double} if the method body's feature
return type is {\tt real[]}.  Otherwise, it can evaluate to anything - even an
object - and the resulting value will be converted to a {\tt String}. \\

The syntax of a {\tt sense} statement in a feature generator is \\

\vspace{-.25cm}
{\tt sense \emph{expression} :{\tiny{ }}\emph{expression};} \\
\vspace{-.25cm}

\noindent
The first expression may evaluate to anything.  Its {\tt String} value will be
appended to the name of the method body to create the name of the feature.
The second expression will be interpreted as that feature's value.  It must
evaluate to a {\tt double} if the method body's feature return type is {\tt
real\%}.  Otherwise, it can evaluate to anything and the resulting value will
be converted to a {\tt String}. \\

The single expression form of the {\tt sense} statement may also appear in a
feature generator method body.  In this case, the expression represents the
feature's name, and that feature is assumed to be Boolean with a value of {\tt
true}.

\subsubsection{Invoking Classifiers} \label{subsection:invokingClassifiers}
Under the right circumstances, any classifier may be invoked inside an LBJ
method body just as if it were a method.  The syntax of a classifier
invocation is simply {\tt \emph{name}(\emph{object})}, where
{\tt\emph{object}} is the object to be classified and {\tt\emph{name}} follows
the same rules as when a classifier is named in a classifier expression (see
Section \ref{subsection:classifierNames}).  In general, the semantics of such
an invocation are such that the value(s) and not the names of the produced
features are returned at the call site.

More specifically:

\vspace{-.15cm}
\begin{itemize}
\item
A classifier defined to return exactly one feature may be invoked anywhere
within a method body.  If it has feature return type {\tt discrete}, a {\tt
String} will be returned at the call site.  Otherwise, a {\tt double} will be
returned.

\item
Classifiers defined to return an array of features may also be invoked
anywhere within a method body.  Usually, they will return either {\tt
String[]} or {\tt double[]} at the call site when the classifier has feature
return type {\tt discrete[]} or {\tt real[]} respectively.  The only exception
to this rule is discussed next.

\item
When a {\tt sense} statement appears in a method body defined to return an
array of features, the lone argument to that {\tt sense} statement may be an
invocation of another array returning classifier of the same feature return
type.  In this case, all of the features returned by the invoked classifier
are returned by the invoking classifier, renamed to take the invoking
classifier's name and indexes.

\item
Feature generators may only be invoked when that invocation is the entire
expression on the right of the colon in a {\tt sense} statement contained in
another feature generator of the same feature return type\footnote{Any feature
generator may be invoked in this context in a classifier whose feature return
type is {\tt mixed\%}.}.  In this case, this single {\tt sense} statement will
return every feature produced by the invoked generator with the following
modification.  The name of the containing feature generator and the {\tt
String} value of the expression on the left of the colon are prepended to
every feature's name.  Thus, an entire set of features can be translated to
describe a different context with a single {\tt sense} statement.
\end{itemize}

\subsubsection{Syntax Limitations}
When the exact computation is known, LBJava intends to allow the programmer to
explicitly define a classifier using arbitrary Java.  However, the current
version of LBJava suffers from one major limitation.  All J2SE 1.4.2 statement
and expression syntax is accepted, excluding class and interface definitions.
In particular, this means that anonymous classes currently cannot be defined
or instantiated inside an LBJava method body.

\section{Constraints} \label{section:constraints}

Many modern applications involve the repeated application of one or more
learning classifiers in a coordinated decision making process.  Often, the
nature of this decision making process restricts the output of each learning
classifier on a call by call basis to make all these outputs coherent with
respect to each other.  For example, a classification task may involve
classifying some set of objects, at most one of which is allowed to take a
given label.  If the learned classifier is left to its own devices, there is
no guarantee that this constraint will be respected.  Using LBJ's constraint
and inference syntax, constraints such as these are resolved automatically in
a principled manner. \\

More specifically, Integer Linear Programming (ILP) is applied to resolve the
constraints such that the expected number of correct predictions made by each
learning classifier involved is maximized.  The details of how ILP works are
beyond the scope of this user's manual.  See\\ \cite{PunyakanokRoYi08} for
more details. \\

This section covers the syntax and semantics of constraint declarations and
statements.  However, simply declaring an LBJava constraint has no effect on the
classifiers involved.  Section \ref{section:inference} introduces the syntax
and semantics of LBJava inference procedures, which can then be invoked (as
described in Section \ref{subsubsection:inferenceInvocations}) to produce new
classifiers that respect the constraints.

\subsection{Constraint Statements} \label{subsection:constraintStatements}

LBJava constraints are written as arbitrary first order Boolean logic expressions
in terms of learning classifiers and the objects in a Java application.  The
LBJava constraint statement syntax is parameterized by Java expressions, so that
general constraints may be expressed in terms of the objects of an internal
representation whose exact shape is not known until run-time.  The usual
operators and quantifiers are provided, as well as the {\tt atleast} and {\tt
atmost} quantifiers, which are described below.  The only two predicates in
the constraint syntax are equality and inequality (meaning string comparison),
however their arguments may be arbitrary Java expressions (which will be
converted to strings). \\

Each declarative constraint statement contains a single constraint expression
and ends in a semicolon.  Constraint expressions take one of the following
forms:

\vspace{-.2cm}
\begin{itemize}
\item An equality predicate \hspace{.05cm}
      {\tt\emph{Java-expression}\hspace{.1cm}::\hspace{.1cm}\emph{Java-expression}}
\item An inequality predicate \hspace{.05cm}
      {\tt\emph{Java-expression}\hspace{.1cm}!:\hspace{.1cm}\emph{Java-expression}}
\item A constraint invocation \hspace{.05cm}
      {\tt@\emph{name}(\emph{Java-expression})} \\
      where the expression must evaluate to an object and {\tt\emph{name}}
      follows similar rules as classifier names when they are invoked.  In
      particular, if {\tt MyConstraint} is already declared in {\tt
      SomeOtherPackage}, it may be invoked with {\tt
      @SomeOtherPackage.MyConstraint(object)}.
\item The negation of an LBJava constraint \hspace{.05cm}
      {\tt !\emph{constraint}}
\item The conjunction of two LBJava constraints \hspace{.05cm}
      {\tt\emph{constraint} \verb|/\| \emph{constraint}}
\item The disjunction of two LBJava constraints \hspace{.05cm}
      {\tt\emph{constraint} \verb|\/| \emph{constraint}}
\item An implication \hspace{.05cm}
      {\tt\emph{constraint} => \emph{constraint}}
\item The equivalence of two LBJava constraints \hspace{.05cm}
      {\tt\emph{constraint} <=> \emph{constraint}}
\item A universal quantifier \hspace{.05cm}
      {\tt forall (\emph{type name} in \emph{Java-expression})
        \emph{constraint}} \\
      where the expression must evaluate to a Java {\tt Collection} containing
      objects of the specified type, and the constraint may be written in
      terms of {\tt\emph{name}}.
\item An existential quantifier \hspace{.05cm}
      {\tt exists (\emph{type name} in \emph{Java-expression})
        \emph{constraint}}
\item An ``at least'' quantifier \\
      \mbox{\hspace{.4cm}}
      {\tt atleast \emph{Java-expression} of (\emph{type name} in
          \emph{Java-expression}) \emph{constraint}} \\
      where the first expression must evaluate to an an {\tt int}, and the
      other parameters play similar roles to those in the universal
      quantifier.
\item An ``at most'' quantifier \\
      \mbox{\hspace{.4cm}}
      {\tt atmost \emph{Java-expression} of (\emph{type name} in
          \emph{Java-expression}) \emph{constraint}}
\end{itemize}

\noindent
Above, the operators have been listed in decreasing order of precedence.  Note
that this can require parentheses around quantifiers to achieve the desired
effect.  For example, the conjunction of two quantifiers can be written like
this: \\

\vspace{-.25cm}
{\tt (exists (Word w in sentence) someLearner(w) ::\hspace{.2cm}"good")}

\hspace{.2cm}
{\tt \verb|/\| (exists (Word w in sentence)
                  otherLearner(w) ::\hspace{.2cm}"better")} \\
\vspace{-.25cm}

The arguments to the equality and inequality predicates are treated specially.
If any of these arguments is an invocation of a learning classifier, that
classifier and the object it classifies become an inference variable, so that
the value produced by the classifier on that object is subject to change by
the inference procedure.  The values of all other expressions that appear as
an argument to either type of predicate are constants in the inference
procedure.  This includes, in particular, expressions that include a learning
classifier invocation as a subexpression.  These learning classifier
invocations are not treated as inference variables.


\subsection{Constraint Declarations} \label{subsection:constraintDeclarations}

An LBJava constraint declaration declares a Java method whose purpose is to
locate the objects involved in the inference and generate the constraints.
Syntactically, an LBJava constraint declaration starts with a header indicating
the name of the constraint and the type of object it takes as input, similar
to a method declaration with a single parameter: \\

\vspace{-.25cm}
{\tt constraint \emph{name}(\emph{type name}) \emph{method-body}} \\
\vspace{-.25cm}

\noindent
where {\tt\emph{method-body}} may contain arbitrary Java code interspersed
with constraint statements all enclosed in curly braces.  When invoked with
the {\tt @} operator (discussed in Section
\ref{subsection:constraintStatements}), the occurrence of a constraint
statement in the body of a constraint declaration signifies not that the
constraint expression will be evaluated in place, but instead that a first
order representation of the constraint expression will be constructed for an
inference algorithm to manipulate.  The final result produced by the
constraint is the conjunction of all constraint statements encountered while
executing the constraint. \\

In addition, constraints declared in this way may also be used as Boolean
classifiers as if they had been declared: \\

\vspace{-.25cm}
{\tt discrete\{"false", "true"\} \emph{name}(\emph{type name})} \\
\vspace{-.25cm}

\noindent
Thus, a constraint may be invoked as if it were a Java method (i.e., without
the {\tt @} symbol described in Section \ref{subsection:constraintStatements})
anywhere in an LBJava source file, just like a classifier.  Such an invocation
will evaluate the constraint in place, rather than constructing its first
order representation.

\section{Inference} \label{section:inference}

The syntax of an LBJava inference has the following form: \\

\vspace{-.25cm}
{\tt inference \emph{name} head \emph{type name}}

{\tt \{}

\begin{tabular}{ll}
\hspace{.2cm} {\tt $[$\emph{type name method-body}$]+$} &
{\tt // "Head-finder" methods} \\
\hspace{.2cm}
{\tt $[[$\emph{name}$]$ normalizedby \emph{name}\hspace{.1cm};$]*$} &
{\tt // How to normalize scores} \\
\hspace{.2cm} {\tt subjectto \emph{method-body}} &
{\tt // Constraints} \\
\hspace{.2cm} {\tt with \emph{instance-creation-expression}} &
{\tt // Names the algorithm}
\end{tabular}

{\tt \}} \\
\vspace{-.25cm}

\noindent
This structure manages the functions, run-time objects, and constraints
involved in an inference.  Its header indicates the name of the inference and
its \emph{head} parameter.  The head parameter (or head object) is an object
from which all objects involved in the inference can be reached at run-time.
This object need not have the same type as the input parameter of any learned
function involved in the inference.  It also need not have the same type as
the input parameter of any constraint involved in the inference, although it
often will. \\

After the header, curly braces surround the body of the inference.  The body
contains the following four elements.  First, it contains at least one ``head
finder'' method.  Head finder methods are used to locate the head object given
an object involved in the inference.  Whenever the programmer wishes to use
the inference to produce the constrained version of a learning classifier
involved in the inference, that learning classifier's input type must have a
head finder method in the inference body.  Head finder methods are usually
very simple.  For example: \\

\vspace{-.25cm}
{\tt Word w \{ return w.getSentence(); \}} \\
\vspace{-.25cm}

\noindent
might be an appropriate head finder method when the head object has type {\tt
Sentence} and one of the classifiers involved in the inference takes {\tt
Word}s as input. \\

Second, the body specifies how the scores produced by each learning classifier
should be normalized.  The LBJava library contains a set of normalizing functions
that may be named here.  It is not strictly necessary to use normalization
methods, but doing so ensures that the scores computed for each possible
prediction may be treated as a probability distribution by the inference
algorithm.  Thus, we may then reason about the inference procedure as
optimizing the expected number of correct predictions. \\

The syntax of normalizer clauses enables the programmer to specify a different
normalization method for each learning classifier involved in the inference.
It also allows for the declaration of a default normalizer to be used
by learning classifiers which were not given normalizers individually.  For
example: \\

\vspace{-.25cm}
{\tt SomeLearner normalizedby Sigmoid;}

{\tt normalizedby Softmax;} \\
\vspace{-.25cm}

\noindent
These normalizer clauses written in any order specify that the {\tt
SomeLearner} learning classifier should have its scores normalized with the
{\tt Sigmoid} normalization method and that all other learning classifiers
involved in the inference should be normalized by {\tt Softmax}.

Third, the {\tt subjectto} clause is actually a constraint declaration
(see Section \ref{section:constraints}) whose input parameter is the head
object.  For example, let's say an inference named {\tt MyInference} is
declared like this: \\

\vspace{-.25cm}
{\tt inference MyInference head Sentence s} \\
\vspace{-.25cm}

\noindent
and suppose also that several other constraints have been declared named
(boringly) {\tt Constraint1}, {\tt Constraint2}, and {\tt Constraint3}.
Then an appropriate {\tt subjectto} clause for {\tt MyInference} might look
like this: \\

\vspace{-.25cm}
{\tt subjectto
     \{ \verb|@Constraint1(s) /\ @Constraint2(s) /\ @Constraint3(s);| \}} \\
\vspace{-.25cm}

\noindent
The {\tt subjectto} clause may also contain arbitrary Java, just like any
other constraint declaration. \\

Finally, the {\tt with} clause specifies which inference algorithm to use.  It
functions similarly to the {\tt with} clause of a learning classifier
expression (see Section \ref{subsection:LCE}).

\section{``Makefile'' Behavior}

An LBJava source file also functions as a makefile in the following sense.
First, code will only be generated for a classifier definition when it is
determined that a change has been made\footnote{When the file(s) containing
the translated code for a given classifier do not exist, this is, of course,
also interpreted as a change having been made.} in the LBJava source for that
classifier since the last time the compiler was executed.  Second, a learning
classifier will only be trained if it is determined that the changes made
affect the results of learning.  More precisely, any classifier whose
definition has changed lexically is deemed ``affected''.  Furthermore, any
classifier that makes use of an affected classifier is also affected.  This
includes method bodies that invoke affected classifiers and conjunctions and
learning classifiers involving at least one affected classifier.  A learning
classifier will be trained if and only if a change has been made to its own
source code or it is affected.  Thus, when an LBJava source contains many
learning classifiers and a change is made, time will not be wasted re-training
those that are unaffected. \\

In addition, the LBJava compiler will automatically compile any Java source files
that it depends on, so long as the locations of those source files are
indicated with the appropriate command line parameters (see Section
\ref{section:commandLine}).  For example, if the classifiers in an LBJava source
file are defined to take classes from the programmer's internal representation
as input, the LBJava compiler will automatically compile the Java source files
containing those class' implementations if their class files don't already
exist or are out of date.

