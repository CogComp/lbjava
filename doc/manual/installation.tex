
\chapter{Installation and Command Line Usage}

\section{Installation} \label{section:installation}

LBJava is written entirely in Java - almost.  The Java Native Interface (JNI) is
utilized to interface with the GNU Linear Programming Kit (GLPK) which is used
to perform inference (see Section \ref{subsection:GLPK}), requiring a small
amount of C to complete the connection.  This C code must be compiled as a
library so that it can be dynamically linked to the JVM at run-time in any
application that uses inference.  Thus, the GNU Autotools became a natural
choice for LBJava's build system.  More information on building and installing
LBJava from its source code is presented below. \\

On the other hand, some users' applications may not require LBJava's automated
inference capabilities.  In this case, installation is as easy as downloading
a jar file from the Cognitive Computation Group's website\footnote{{\tt
http://cogcomp.cs.illinois.edu/}} and adding it to your {\tt CLASSPATH}
environment variable. If this is
your chosen method of installation, you may safely skip to the section on
command line usage below. \\

Alternatively, the source code for both the compiler and the library can be
downloaded from the same web site.  Download the file {\tt lbj-2.x.x.tar.gz}
and unpack it with the following command: \\

\vspace{-.25cm}
{\tt tar zxf lbj-2.x.x.tar.gz} \\
\vspace{-.25cm}

\noindent
The {\tt lbj-2.x.x} directory is created, and all files in the package are
placed in that directory.  Of particular interest is the file {\tt configure}.
This is a shell script designed to automatically detect pertinent parameters
of your system and to create a set of makefiles that builds LBJava with respect
to those parameters.  In particular, this script will detect whether or not
you have GLPK installed.  If you do, LBJava will be compiled with inference
enabled.\footnote{GLPK is a separate software package that must be downloaded,
compiled, and installed before LBJava is configured in order for LBJava to make use
of it.  Download it from {\tt http://www.gnu.org/software/glpk/}}  The {\tt
configure} script itself was built automatically by the GNU Autotools, but you
will \emph{not} need them installed on your system to make use of it. \\

By default, the {\tt configure} script will create makefiles that intend to
install LBJava's JNI libraries and headers in system directories such as {\tt
/usr/local/lib} and {\tt /usr/local/include}.  If you have root privileges on
your system, this will work just fine.  Otherwise, it will be necessary to use
{\tt configure}'s {\tt --prefix} command line option.  For example, running
{\tt configure} with {\tt --prefix=\$HOME} will create makefiles that install
LBJava's libraries and headers in similarly named subdirectories of your user
account's root directory, such as \verb|~/lib| and \verb|~/include|.  The {\tt
configure} script has many other options as well.  Use {\tt --help} on the
command line for more information. \\

If you choose to use the {\tt --prefix} command line option, then it is a
reasonable assumption that you also used it when building and installing GLPK.
In that case, the following environment variables must be set \emph{before}
running LBJava's {\tt configure} script.  {\tt CPPFLAGS} is used to supply
command line parameters to the C preprocessor.  We will use it to add the
directory where the GLPK headers were installed to the include path.  {\tt
LDFLAGS} is used to supply command line parameters to the linker.  We will use
it to add the directory where the GLPK library was installed to the list of
paths that the linker will search in.  For example, in the {\tt bash} shell:
\\

\vspace{-.25cm}
{\tt export CPPFLAGS=-I\$HOME/include}

{\tt export LDFLAGS=-L\$HOME/lib} \\
\vspace{-.25cm}

\noindent
or in {\tt csh}: \\

\vspace{-.25cm}
{\tt setenv CPPFLAGS} \verb|-I${HOME}/include|

{\tt setenv LDFLAGS} \verb|-L${HOME}/lib| \\
\vspace{-.25cm}

The last step in making sure that inference will be enabled is to ensure that
the file {\tt jni.h} is on the include path for the C preprocessor.  This file
comes with your JVM distribution.  It is often installed in a standard
location already, but if it isn't, we must set {\tt CPPFLAGS} in such a way
that it adds all the paths we are interested in to the include path.  For
example, in the {\tt bash} shell: \\

\vspace{-.25cm}
{\tt export JVMHOME=/usr/lib/jvm/java-6-sun}

{\tt export CPPFLAGS="\$CPPFLAGS -I\$JVMHOME/include"}

{\tt export CPPFLAGS="\$CPPFLAGS -I\$JVMHOME/include/linux"} \\
\vspace{-.25cm}

\noindent
or in {\tt csh}: \\

\vspace{-.25cm}
{\tt setenv JVMHOME /usr/lib/jvm/java-6-sun}

{\tt setenv CPPFLAGS} \verb|"${CPPFLAGS} -I${JVMHOME}/include"|

{\tt setenv CPPFLAGS} \verb|"${CPPFLAGS} -I${JVMHOME}/include/linux"| \\
\vspace{-.25cm}

\noindent
At long last, we are ready to build and install LBJava with the following
command: \\

\vspace{-.25cm}
\verb|./configure --prefix=$HOME && make && make install| \\
\vspace{-.25cm}

\noindent
If all goes well, you will see a message informing you that a library has been
installed and that certain extra steps may be necessary to ensure that this
library can be used by other programs.  Follows these instructions.  Also,
remember to add the {\tt lbj-2.x.x} directory to your {\tt CLASSPATH}
environment variable. \\

LBJava's makefile also contains rules for creating the jars that are separately
downloadable from the website and for creating the Javadoc documentation for
both compiler and library.  To create the jars, simply type {\tt make jars}.
To create the Javadoc documentation, you must first set the environment
variable {\tt LBJ2\_DOC} equal to the directory in which you would like the
documentation created.  Then type {\tt make doc}. \\

Finally, users of the VIM editor may be interested in {\tt lbjava.vim}, the LBJava
syntax highlighting file provided in the tar ball.  If you have not done so
already, create a directory named {\tt .vim} in your home directory.  In that
directory, create a file named {\tt filetype.vim} containing the following
text: \\

\vspace{-.25cm}
\verb|if exists("did_load_filetypes")|

\verb|  finish|

\verb|endif|

\verb|augroup filetypedetect|

\verb|  au! BufRead,BufNewFile *.lbj         setf lbj|

\verb|augroup END| \\
\vspace{-.25cm}

\noindent
Then create the subdirectory {\tt .vim/syntax} and place the provided {\tt
lbjava.vim} file in that subdirectory.  Now, whenever VIM edits a file whose
extension is {\tt .lbj}, LBJava syntax highlighting will be enabled.

\newpage
\section{Command Line Usage} \label{section:commandLine}

The LBJava compiler is itself written in Java.  It calls {\tt javac} both to
compile classes that its source file depends on and to compile the code it
generates.  Its command line usage is as follows: \\

\vspace{-.25cm}
{\tt java lbjava.Main [options] <source file>} \\
\vspace{-.25cm}

\noindent
where {\tt [options]} is zero or more of the following:

\begin{center}
\begin{tabular}{ll}
{\tt -c} &
Compile only: This option tells LBJ2 to translate the given source to \\
& Java, but not to compile the generated Java sources or do any training.
\vspace{.2cm} \\

{\tt -d <dir>} &
Any class files generated during compilation will be written in the \\
& specified directory, just like javac's -d command line parameter.
\vspace{.2cm} \\

{\tt -j <a>} &
Sends the contents of {\tt <a>} to {\tt javac} as command line arguments while
\\
& compiling.  Don't forget to put quotes around {\tt <a>} if there is more
than \\
& one such argument or if the argument has a parameter. \vspace{.2cm} \\

{\tt -t <n>} &
Enables progress output during training of learning classifiers.  A \\
& message containing the date and time will be printed to {\tt STDOUT} after
\\
& every {\tt <n>} training objects have been processed. \vspace{.2cm} \\

{\tt -v} &
Prints the version number and exits. \vspace{.2cm} \\

{\tt -w} &
Disables the output of warning messages. \vspace{.2cm} \\

{\tt -x} &
Clean: This option deletes all files that would have been generated \\
& otherwise.  No new code is generated, and no training takes place.
\vspace{.2cm} \\

{\tt -gsp <dir>} &
LBJava will potentially generate many Java source files.  Use this option to \\
& have LBJava write them to the specified directory instead of the current \\
& directory.  {\tt <dir>} must already exist.  Note that LBJava will also compile
\\
& these files which can result in even more class files than there were \\
& sources.  Those class files will also be written in {\tt <dir>} unless the
{\tt -d} \\
& command line parameter is utilized as well. \vspace{.2cm} \\

{\tt -sourcepath <dir>} &
If the LBJava source depends on classes whose source files cannot be found \\
& on the user's classpath, specify the directories where they can be found \\
& using this parameter. It works just like javac's {\tt -sourcepath} command
\\
& line parameter.  \vspace{.2cm} \\

{\tt --parserDebug} &
Debug: This option enables debugging output during parsing. \vspace{.2cm} \\

{\tt --lexerOutput} &
Lexer output: With this option enabled, the lexical token stream will be \\
& printed, after which the compiler will terminate. \vspace{.2cm} \\

{\tt --parserOutput} &
Parser output: With this option enabled, the parsed abstract syntax \\
& tree will be printed, after which the compiler will quit. \vspace{.2cm} \\

{\tt --semanticOutput} &
Semantic analysis output: With this option enabled, some information \\
& computed by semantic analysis will be printed, after which the compiler \\
& will quit.
\end{tabular}
\end{center}

\newpage
By default, all files generated by LBJava will be created in the same directory
in which the LBJava source file is found.  To place generated Java sources in a
different directory, use the {\tt -gsp} (or {\tt -generatedsourcepath})
command line option.  The lexicon and example files described in Section
\ref{subsection:LCE} are also placed in the directory specified by this
option.  In addition, the generated sources' class files will be created in
that directory unless the {\tt -d} command line option is also specified.
This option places all generated class files in the specified directory, just
like {\tt javac}'s {\tt -d} option.  The ``learning classifier'' file with
extension {\tt .lc} (also discussed in Section \ref{subsection:LCE}) will also
be placed in the directory specified by the {\tt -d} option.  Another option
similar to {\tt javac} is the {\tt -sourcepath} option for specifying extra
directories in which Java source files are found.  Both the {\tt -d} and {\tt
-sourcepath} options should be given directly to LBJava if they are given at all.
Do not specify them inside LBJava's {\tt -j} option.  Finally, LBJava does not offer
a {\tt -classpath} option.  Simply give this parameter to the JVM instead. \\

For example, say an employee of the XYZ company is building a new software
package called ABC with the help of LBJava.  This is a large project, and
compiling the LBJava source file will generate many new Java sources.  She places
her LBJava source file in a new working directory along side three new
subdirectories: {\tt src}, {\tt class}, and {\tt lbj}. \\

\vspace{-.25cm}
{\tt \$ ls}

\verb|abc.lbj  src/    class/  lbj/| \\
\vspace{-.25cm}

\noindent
Next, since all the source files in the ABC application will be part of the
{\tt com.xyz.abc} package, she creates the directory structure {\tt
com/xyz/abc} as a subdirectory of the {\tt src} directory.  Application source
files are then placed in the {\tt src/com/xyz/abc} directory.  Next, at the
top of her LBJava source file she writes the line {\tt package com.xyz.abc;}.
Now she is ready to run the following commands: \\

\vspace{-.25cm}
{\tt \$ java -cp \$CLASSPATH:class lbjava.Main -sourcepath src -gsp lbj -d class
  abc.lbj}

. . .

{\tt \$ javac -classpath \$CLASSPATH:class -sourcepath lbj:src -d class}
\verb|\| \\
\verb|           src/com/xyz/abc/*.java|

{\tt \$ jar cvf abc.jar -C class com} \\
\vspace{-.25cm}

\noindent
The first command creates the {\tt com/xyz/abc} directory structure in both of
the {\tt lbj} and {\tt class} directories.  LBJava then generates new Java
sources in the {\tt lbj/com/xyz/abc} directory and class files in the {\tt
class/com/xyz/abc} directory.  Now that the necessary classifiers'
implementations exist, the second command compiles the rest of the
application.  Finally, the last command prepares a jar file containing the
entire ABC application.  Users of ABC need only add {\tt abc.jar} to their
{\tt CLASSPATH}. \\

There are two other JVM command line parameters that will be of particular
interest to programmers working with large datasets.  Both increase the amount
of memory that Java is willing to utilize while running.  The first is {\tt
-Xmx<size>} which sets the maximum Java heap size.  It should be set as high
as possible, but not so high that it causes page-faults for the JVM or for
some other application on the same computer.  This value must be a multiple of
1024 greater than 2MB and can be specified in kilobytes ({\tt K}, {\tt k}),
megabytes ({\tt M}, {\tt m}), or gigabytes ({\tt G}, {\tt g}). \\

The second is {\tt -XX:MaxPermSize=<size>} which sets the maximum size of the
\emph{permanent generation}.  This is a special area of the heap which stores,
among other things, canonical representations for the {\tt String}s in a Java
application.  Since a learned classifier can contain many {\tt String}s, it
may be necessary to set it higher than the default of 64 MB.  For more
information about the heap and garbage collection, see {\tt
http://java.sun.com/docs/hotspot/gc5.0/gc\_tuning\_5.html}. \\

With these two command line parameters, a typical LBJava compiler command line
might look like: \\

\vspace{-.25cm}
{\tt java -Xmx512m -XX:MaxPermSize=512m lbjava.Main Test.lbj} \\
\vspace{-.25cm}

\noindent
When it is necessary to run the compiler with these JVM settings, it will also
be necessary to run the application that uses the generated classifiers with
the same or larger settings.

